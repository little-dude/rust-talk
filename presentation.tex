\documentclass{beamer}

\usepackage{hyperref}
\hypersetup{colorlinks=true, linkcolor=blue, urlcolor=blue, citecolor=blue, anchorcolor=blue}
\usepackage{minted}
\usepackage{multicol} % https://tex.stackexchange.com/a/225691/32098

\usetheme{bjeldbak}
\setbeamertemplate{footline}{
	\begin{beamercolorbox}[colsep=1.25pt]{lower separation line foot}
	\hfill\scriptsize{\insertframenumber/\inserttotalframenumber}
	\end{beamercolorbox}
}
\setbeamercovered{invisible}

\AtBeginSection[]{
    \begin{frame}
        \vfill
        {\footnotesize{\tableofcontents[currentsection, hideothersubsections]}}
        \vfill
    \end{frame}
}

\AtBeginSubsection[]{
    \begin{frame}
        \vfill
        \begin{center}
            \Large{\insertsubsection\par}
        \end{center}
        \vfill
    \end{frame}
}

\begin{document}

{%
	% enleve l'entete de la page de titre
	\setbeamertemplate{headline}{}
		
	% enleve le pied de page 
	\setbeamertemplate{footline}{
		\begin{beamercolorbox}[colsep=1.25pt]{lower separation line foot}
		\end{beamercolorbox}} 
    \title{Rust}
    \author{Corentin Henry}
    \institute{Nuage Networks}    
    \date{February, 2018}
    \frame{\titlepage}
}
\addtocounter{framenumber}{-2}

\begin{frame}
    \begin{center}
        \texttt{Rust} is a systems programming language that runs blazingly fast, prevents segfaults, and guarantees thread safety. 
    \end{center}
\end{frame}


\begin{frame}
    \begin{itemize}
        \item zero-cost abstractions
        \item \fbox{guaranteed memory safety}
        \item \fbox{threads without data races}
        \item trait-based generics
        \item pattern matching
        \item type inference
        \item minimal runtime
        \item efficient C bindings
    \end{itemize}
\end{frame}

\section{Memory safety}

\begin{frame}
    \frametitle{What is memory safety?}
    \begin{itemize}
        \item buffer overflow, over-read
        \item use after free, double free, invalid free
        \item race conditions
        \item dangling pointer dereference
        \item stack exhaustion
        \item heap exhaustion
    \end{itemize}
\end{frame}

\begin{frame}
    \begin{multicols}{2}
        \inputminted[fontsize=\tiny]{c}{code/vec_no_comment.c}
    \end{multicols}
\end{frame}

\begin{frame}
    \begin{multicols}{2}
        \inputminted[fontsize=\tiny]{c}{code/vec.c}
    \end{multicols}
\end{frame}

\begin{frame}
    \frametitle{Mitigations: C++}
    Smart pointers, RAII, move semantics: \texttt{unique\_ptr},
    \texttt{shared\_ptr}, \texttt{weak\_ptr}

    \begin{itemize}
        \item control
        \item zero cost abstractions
    \end{itemize}
\end{frame}

\begin{frame}
    \frametitle{Good but not enough}
    \inputminted[]{c++}{code/invalid_ref.cpp}
\end{frame}

\begin{frame}
    \frametitle{Good but not enough}
    \inputminted[]{c++}{code/invalid_ref_commented.cpp}
\end{frame}

\begin{frame}
    \frametitle{What about GC?}
    \begin{itemize}
        \item performance cost
        \item memory usage cost
        \item loss of control
        \item does not everything: data races, iterator invalidation, NPEs, etc.
    \end{itemize}
\end{frame}

\section{Ownership and Borrowing}

\subsection{Ownership}

\begin{frame}
    \inputminted[fontsize=\scriptsize]{rust}{code/ownership.rs}
\end{frame}

\begin{frame}
    \inputminted[fontsize=\scriptsize]{rust}{code/ownership2.rs}
\end{frame}

\begin{frame}
    \inputminted[fontsize=\scriptsize]{rust}{code/ownership3.rs}
\end{frame}

\begin{frame}
    \begin{center}
        What if we want to use the data after?
    \end{center}
\end{frame}

\subsection{Borrowing}

\begin{frame}
    \begin{center}
        Shared borrows (\texttt{\&T})

        \vspace{3em}

        Mutable borrows (\texttt{\&mut T})
    \end{center}
\end{frame}

\begin{frame}
    \begin{center}
        Shared borrow (\texttt{\&T}):

        \vspace{3em}
        
        multiple read-only references to data.
    \end{center}
\end{frame}

\begin{frame}
    \frametitle{Shared borrow}
    \inputminted[fontsize=\scriptsize]{rust}{code/shared_borrow.rs}
\end{frame}

\begin{frame}
    \begin{center}
        Mutable borrow (\texttt{\&mut T})
        
        \vspace{3em}
        
        Single read/write reference to data.
    \end{center}
\end{frame}

\begin{frame}
    \frametitle{Mutable borrow (1)}
    \inputminted[fontsize=\scriptsize]{rust}{code/mutable_borrow1.rs}
\end{frame}

\begin{frame}
    \frametitle{Mutable borrow (2)}
    \inputminted[fontsize=\scriptsize]{rust}{code/mutable_borrow2.rs}
\end{frame}

\section{Concurrency}

\begin{frame}
    \begin{itemize}
        \frametitle{Concurrency Pitfalls}
        \item resource starvation (deadlocks, livelocks)
        \item data races:
            \begin{itemize}
                \item two (or more) threads access the same memory concurrently
                \item at least one is writing
                \item there is no synchronisation
            \end{itemize}
    \end{itemize}
\end{frame}

\begin{frame}
    \frametitle{Data race recipe}
    \begin{center}
        Mutation + Aliasing + No ordering
    \end{center}
\end{frame}

% \begin{frame}
%     \begin{center}
%         No data race <=> No accidentally shared state
%     \end{center}
% \end{frame}

\begin{frame}
    \frametitle{Concurrency paradigms}
    \begin{itemize}
        \item isolated message passing
        \item split-join
        \item shared state
    \end{itemize}
\end{frame}

\section{Fearless concurrency}


\begin{frame}
    \frametitle{Isolated message passing rules}
    \begin{itemize}
        \item threads communicate through channels
        \item no shared state: do not acces data after sending it
    \end{itemize}
\end{frame}

\begin{frame}
    \frametitle{Enforcing rules with Ownership}
    \inputminted[fontsize=\tiny]{rust}{code/isolated_message_passing.rs}
\end{frame}

\begin{frame}
    Rust captures at compile time, the discipline of message passing programming
\end{frame}

\begin{frame}
    \frametitle{Shared State rules}
    Share data, but synchronise accesses:
    \begin{itemize}
        \item only one thread can access the data (lock)
        \item \texttt{N} threads can \textbf{read} the data \textit{or} 1 thread can \textbf{write} it (rwlock)
    \end{itemize}
\end{frame}

\begin{frame}
    \frametitle{Shared State with Borrowing + Ownership}
        \begin{itemize}
            \item no thread owns the data
            \item \texttt{Mutex<T>} owns the data
            \item threads share a non-mutable reference to the lock (borrowing)
            \item only one way to access the data: locking the lock
        \end{itemize}
        \vspace{1.5em}
        \inputminted{rust}{code/lock.rs}
\end{frame}

\begin{frame}
    \frametitle{Shared State: Borrowing + Ownership}
    Rust captures at compile time, the discipline of lock programming.
    % cannot forget to lock => no other way to access the data
    % cannot forget to unlock
\end{frame}

\begin{frame}
    \frametitle{A word about thread safety}
    \begin{itemize}
        \item thread safety is encoded in the type system and statically checked
        \item \texttt{Send} marker: it is safe to send the type to another thread
        \item \texttt{Sync} marker: it is safe to have multiple read-only references of the type between threads (\texttt{\&T} is \texttt{Send})
    \end{itemize}
\end{frame}



% \begin{frame}
%     \begin{center}
%         Rust type system
%     \end{center}
% \end{frame}
% 
% \begin{frame}
% 
%     A set of rules that assigns a property called type to the various constructs program (variables, expressions, functions...).
% 
%     \begin{itemize}
%         \item correctness: checking semantics (\mintinline{python}{1 / "foo"})
%         \item safety
%         \item abstractions (string vs array of bytes)
%         \item documentation (timestamp type instead of integer)
%         \item compiler optimizations (memory alignment)
%     \end{itemize}
% 
% \end{frame}
% 
% \begin{frame}
%     \frametitle{Type safety (1)}
%         \begin{center}
%             "Well typed" programs cannot go wrong
%         \end{center}
% \end{frame}
% 
% \begin{frame}
%     \frametitle{Type safety (2)}
%         \begin{itemize}
%             \item Well typed: respects the rules of the type system
%             \item Going wrong: exhibiting undefined behavior (UB)
%         \end{itemize}
% \end{frame}
% 
% \begin{frame}
%     \frametitle{C/C++ are not type-safe}
%     \begin{center}
%         \begin{minipage}[t]{0.6\textwidth}
%             \inputminted[fontsize=\scriptsize]{c}{code/type_unsafe.c}
%             \vspace{1em}
%             \hline
%             \vspace{1em}
%             \inputminted[fontsize=\scriptsize]{cpp}{code/type_unsafe.cpp}
%         \end{minipage}
%     \end{center}
% \end{frame}
% 
% 
% \begin{frame}
%     \begin{center}
%         Abstractions
%     \end{center}
% \end{frame}
% 
% \begin{frame}
%     \begin{itemize}
%         \item Structs and tuples
%         \item Polymorphism:
%             \begin{itemize}
%                 \item Traits
%                 \item Generics
%             \end{itemize}
%         \item \texttt{Enum} (\textit{aka} algebraic data types, \textit{aka} sum types)
%     \end{itemize}
% \end{frame}
% 
% \begin{frame}
%     \frametitle{Structs}
%     \begin{multicols}{2}
%         \inputminted[fontsize=\tiny]{rust}{code/struct.rs}
%     \end{multicols}
% \end{frame}
% 
% \begin{frame}
%     \frametitle{Traits are basically interfaces}
%     \begin{multicols}{2}
%         \inputminted[fontsize=\scriptsize]{rust}{code/trait.rs}
%         \vspace{8em}
%         \begin{itemize}
%             \item traits can be implemented on existing types
%             \item traits must be in scope to be used
%         \end{itemize}
%     \end{multicols}
% \end{frame}

\end{document}
